\documentclass[10pt,leqno ]{article}

\usepackage{amsfonts}
\usepackage{amssymb}
\usepackage{amsmath}
\usepackage{times}
\usepackage{mathtools}
\usepackage{amsthm}
\usepackage{hyperref}
\usepackage[margin=1.5in]{geometry}
\usepackage{setspace}
   
\DeclarePairedDelimiter\set\{\}

\newcommand\customeq[1]{\overset{\mathrm{#1}}{=}}

\newtheorem{theorem}{Theorem}
\theoremstyle{definition} 
\newtheorem{problem}[theorem]{Aufgabe}
     
\newenvironment{solution}[1][L]{\begin{doublespace}\textbf{#1.}\quad }{\ \rule{0.5em}{0.5em}\end{doublespace}}
    
\title{Formale Sprachen und Automaten\\ Hausaufgabe 1}
\author{Marcel Ebert, Pascal Dettmers, Claude (???) \\ TU Berlin}

\begin{document} 
\maketitle 
\vskip .25in 
\thispagestyle{empty}

\begin{problem}
Mengengrundlagen
\end{problem}

\begin{solution}[a]
Gib die mit gelb gekennzeichnete Menge mit nur zwei Mengenoperationen an:

% Falsch wegen mehr als zwei Mengenoperationen (?)
\[ M = (A \cup C) \setminus (((A \cap B) \setminus C) \setminus ((A \cap C) \setminus B)) \]

\end{solution}

\begin{solution}[b]
Berechne: \( (( \set{1,3} \times \set{1} )) \cup \set{1, 3, 1 } \setminus \set{(1,3), 1,2} \)

\begin{align*}
    M &= (( \set{1,3} \times \set{1} )) \cup \set{1, 3, 1 } \setminus \set{(1,3), 1,2} \\
    & \customeq{Def. \times} (\set{(1,1), (3,1)} \cup \set{1,3,1} \setminus \set{(1,3),1,2}) \\
    % hier fehlt vielleicht noch die reduktion von {1,3,1} zu  {1,3} (?)
    & \customeq{Def. \cup} (\set{(1,1),(3,1),1,3} \setminus \set{(1,3),1,2}) \\
    & \customeq{Def. \setminus} \set{(1,1),(3,1),3}
\end{align*}

\end{solution}

\begin{solution}[c]
Berechne: \( ( \set{\emptyset,2} \cup \set{\set{\emptyset}}) \cap \mathcal{P} (\set{ \set{\emptyset}, 2}) \)

\begin{align*}
    % Das P sieht eigentlich anders aus (?)
    M &= ( \set{\emptyset,2} \cup \set{\set{\emptyset}}) \cap \mathcal{P} (\set{ \set{\emptyset}, 2}) \\
    & \customeq{Def. \cup} \set{\emptyset, \set{\emptyset}, 2} \cap \mathcal{P} (\set{ \set{\emptyset}, 2}) \\
    & \customeq{Def. \mathcal{P}} \set{\emptyset, \set{\emptyset}, 2} \cap \set{\emptyset, \set{\set{\emptyset}}, \set{2}, \set{\set{\emptyset}, 2}} \\
    & \customeq{Def. \cap} \set{\emptyset}
\end{align*}

\end{solution}

\begin{problem}
Mengenbeweise
\end{problem}

\begin{solution}[a]
Beweise oder widerlege: Für alle Mengen A und B gilt: \( (A \cap B) \cap A = B \cap A \)

Wir beweisen die Aussage. Seien A, B beliebige Mengen.
\begin{align*}
    & (A \cap B) \cap A = B \cap A \\
    & \customeq{Def. \cap} \set{x \mid x \in \set{y \mid y \in A \land y \in B} \land x \in A} \\
    & \customeq{Def. Komm.} \set{x \mid x \in \set{y \mid y \in B \land y \in A} \land x \in A} \\
    & \customeq{Def. \in} \set{x \mid (x \in B \land x \in A) \land x \in A} \\
    & \customeq{Def. Assoz} \set{x \mid x \in B \land (x \in A \land x \in A)} \\
    & \customeq{Def. Idem. und} \set{x \mid x \in B \land x \in A} \\
    & \customeq{Def. \cap} B \cap A \\
\end{align*}

Somit gilt die Aussage.

\end{solution}

\begin{solution}[b]
Beweise oder widerlege: Für alle Mengen A und B gilt: \( A \cup (A \setminus B ) = A  \)

Wir beweisen die Aussage. Seien A, B beliebige Mengen.
\begin{align*}
    & A \cup (A \setminus B ) = A \\
    & \customeq{Def. \cup} \set{x \mid x \in A \lor x \in (A \setminus B)} \\
    & \customeq{Def. \setminus} \set{x \mid x \in A \lor x \in \set{y \mid y \in A \land y \notin B}} \\
    & \customeq{Def. \in} \set{x \mid x \in A \lor (x \in A \land x \notin B)} \\
    & \customeq{Def. Distri.} \set{x \mid (x \in A \lor x \in B) \land (x \in A \land x \in A)} \\
    & \customeq{Def. Idem. oder} \set{x \mid (x \in A \lor x \in B) \land x \in A} \\
    & \customeq{Def. Absorp.} \set{x \mid x \in A} \\
    & \customeq{Def. \in} A \\
\end{align*}

Somit gilt die Aussage.

\end{solution}

\begin{solution}[c]
Beweise oder widerlege: Für alle Mengen A und B gilt: \( (B \cup A) \cap B = A \cap B \)

Wir widerlegen die Aussage durch Angabe eines geeigneten Gegenbeispiels.

Wir wählen \( A \triangleq \set{1,2}, B \triangleq \set{2,3} \).

\begin{align*}
    & (B \cup A) \cap B  \\
    &= (\set{2,3} \cup \set{1,2}) \cap \set{2,3}  \\
    & \customeq{Def. \cup} \set{1,2,3} \cap \set{2,3}  \\
    &= \set{2,3} \\
    & \neq \set{2}  \\
    & \customeq{Def. \cap} \set{1,2} \cap \set{2,3}  \\
    &= A \cap B \\
\end{align*}

Somit gilt die Aussage nicht.

\end{solution}

\end{document}