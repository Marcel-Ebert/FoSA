\documentclass[10pt,leqno ]{article}

\usepackage{amsfonts}
\usepackage{amssymb}
\usepackage{array}
\usepackage{amsmath}
\usepackage{times}
\usepackage{mathtools}
\usepackage{amsthm}
\usepackage{hyperref}
\usepackage[margin=1.5in]{geometry}
\usepackage{setspace}
\usepackage{stmaryrd}

\DeclarePairedDelimiter\set\{\}

\newcommand\customeq[1]{\overset{\mathrm{#1}}{=}}

\newcolumntype{C}{>$c<$}


\newtheorem{theorem}{Theorem}
\theoremstyle{definition} 
\newtheorem{problem}[theorem]{Aufgabe}
     
\newenvironment{solution}[1][L]{\begin{doublespace}\textbf{#1.}\quad }{\ \rule{0.5em}{0.5em}\end{doublespace}}
    
\title{Formale Sprachen und Automaten\\ Abgabe 1}
\author{Marcel Ebert, Pascal Dettmers, Claude (???) \\ TU Berlin}

\begin{document} 
\maketitle 
\vskip .25in 
\thispagestyle{empty}

\begin{problem}
Mengengrundlagen
\end{problem}

\begin{solution}[a]
Gib die mit gelb gekennzeichnete Menge mit nur zwei Mengenoperationen an:

% Falsch wegen mehr als zwei Mengenoperationen (?)
\[ M = (A \setminus B) \triangle C \]

\end{solution}

\begin{solution}[b]
Berechne: \( (( \set{1,3} \times \set{1} )) \cup \set{1, 3, 1 } \setminus \set{(1,3), 1,2} \)

\begin{align*}
    M &= (( \set{1,3} \times \set{1} )) \cup \set{1, 3, 1 } \setminus \set{(1,3), 1,2} \\
    & \customeq{Def. \times} (\set{(1,1), (3,1)} \cup \set{1,3,1} \setminus \set{(1,3),1,2}) \\
    % hier fehlt vielleicht noch die reduktion von {1,3,1} zu  {1,3} (?)
    & \customeq{Def. \cup} (\set{(1,1),(3,1),1,3} \setminus \set{(1,3),1,2}) \\
    & \customeq{Def. \setminus} \set{(1,1),(3,1),3}
\end{align*}

\end{solution}

\begin{solution}[c]
Berechne: \( ( \set{\emptyset,2} \cup \set{\set{\emptyset}}) \cap \mathcal{P} (\set{ \set{\emptyset}, 2}) \)

\begin{align*}
    % Das P sieht eigentlich anders aus (?)
    M &= ( \set{\emptyset,2} \cup \set{\set{\emptyset}}) \cap \mathcal{P} (\set{ \set{\emptyset}, 2}) \\
    & \customeq{Def. \cup} \set{\emptyset, \set{\emptyset}, 2} \cap \mathcal{P} (\set{ \set{\emptyset}, 2}) \\
    & \customeq{Def. \mathcal{P}} \set{\emptyset, \set{\emptyset}, 2} \cap \set{\emptyset, \set{\set{\emptyset}}, \set{2}, \set{\set{\emptyset}, 2}} \\
    & \customeq{Def. \cap} \set{\emptyset}
\end{align*}

\end{solution}

\begin{problem}
Mengenbeweise
\end{problem}

\begin{solution}[a]
Beweise oder widerlege: Für alle Mengen A und B gilt: \( (A \cap B) \cap A = B \cap A \)

Wir beweisen die Aussage. Seien A, B beliebige Mengen.
\begin{align*}
    & & (A \cap B) \cap A = B \cap A \\
    & \customeq{Def. \cap} & \set{x \mid x \in \set{y \mid y \in A \land y \in B} \land x \in A} \\
    & \customeq{Def. Komm.} & \set{x \mid x \in \set{y \mid y \in B \land y \in A} \land x \in A} \\
    & \customeq{Def. \in} & \set{x \mid (x \in B \land x \in A) \land x \in A} \\
    & \customeq{Def. Assoz} & \set{x \mid x \in B \land (x \in A \land x \in A)} \\
    & \customeq{Def. Idem. und} & \set{x \mid x \in B \land x \in A} \\
    & \customeq{Def. \cap} & B \cap A \\
\end{align*}

Somit gilt die Aussage.

\end{solution}

\begin{solution}[b]
Beweise oder widerlege: Für alle Mengen A und B gilt: \( A \cup (A \setminus B ) = A  \)

Wir beweisen die Aussage. Seien A, B beliebige Mengen.
\begin{align*}
    & & A \cup (A \setminus B ) = A \\
    & \customeq{Def. \cup} & \set{x \mid x \in A \lor x \in (A \setminus B)} \\
    & \customeq{Def. \setminus} & \set{x \mid x \in A \lor x \in \set{y \mid y \in A \land y \notin B}} \\
    & \customeq{Def. \in} & \set{x \mid x \in A \lor (x \in A \land x \notin B)} \\
    & \customeq{Def. Distri.} & \set{x \mid (x \in A \lor x \in B) \land (x \in A \land x \in A)} \\
    & \customeq{Def. Idem. oder} & \set{x \mid (x \in A \lor x \in B) \land x \in A} \\
    & \customeq{Def. Absorp.} & \set{x \mid x \in A} \\
    & \customeq{Def. \in} & A \\
\end{align*}

Somit gilt die Aussage.

\end{solution}

\begin{solution}[c]
Beweise oder widerlege: Für alle Mengen A und B gilt: \( (B \cup A) \cap B = A \cap B \)

Wir widerlegen die Aussage durch Angabe eines geeigneten Gegenbeispiels.

Wir wählen \( A \triangleq \set{1,2}, B \triangleq \set{2,3} \).

\begin{align*}
    & & (B \cup A) \cap B  \\
    &= & (\set{2,3} \cup \set{1,2}) \cap \set{2,3}  \\
    & \customeq{Def. \cup} & \set{1,2,3} \cap \set{2,3}  \\
    &= & \set{2,3} \\
    & \neq & \set{2}  \\
    & \customeq{Def. \cap} & \set{1,2} \cap \set{2,3}  \\
    &= & A \cap B \\
\end{align*}

Somit gilt die Aussage nicht.

\end{solution}

\begin{problem}
    Wahrheitstabellen
\end{problem}

\begin{solution}[a]
Beweise oder widerlege nur mit Hilfe einer Wahrheitstabelle oder eines (Gegen-) Beispiels, dass \( \lnot q \land ((r \leftrightarrow (r \rightarrow \bot)) \lor q ) \) kontradiktorisch ist.

\[
\begin{array}{C|C||C|C|C|C|C|C|C|C|C|C}
q & r & $\lnot$ & q & $\overbrace{\land}^{\textbf{\(\downarrow\)}}$ & ((r & $\leftrightarrow$ & (r & $\rightarrow$ & $\bot$)) & $\lor$ & q) \\
\hline
F & F & W & F & F & F & F & F & W & F & F & F \\
F & W & W & F & F & W & F & W & F & F & F & F \\
W & F & F & W & F & F & F & F & W & F & W & W \\
W & W & F & W & F & W & F & W & F & F & W & W \\
\end{array}
\]

Der Hauptjunktor wird immer zu F ausgewertet. Also ist die Formel kontradiktorisch.


\end{solution}

\begin{solution}[b]
Beweise oder widerlege nur mit Hilfe einer Wahrheitstabelle oder eines (Gegen-) Beispiels, dass \( ((s \land \lnot q) \rightarrow r) \lor r \equiv r \lor (s \rightarrow q) \) kontradiktorisch ist.

\end{solution}


\begin{problem}
    Logische Äquivalenz
\end{problem}

\begin{solution}[a]
Gib an: eine Formel, die logisch äquivalent zu \( \bot \) ist und nur  \( \lnot \) und \( \lor \) als Operatoren enthält. 

\( \lnot (q \lor \lnot q) \equiv \bot \)
    
\end{solution}
    
\begin{solution}[b]
Beweise nur mit Hilfe von Äquivalenzumformungen, dass \( q \land (r \rightarrow s) \) und \( \lnot(r \lor \lnot q) \lor (s \land q) \) logisch äquivalent sind.    

\begin{equation*}
    \begin{aligned}
    & & q \land (r \rightarrow s) \\
    & \customeq{Def. Impl.} & q \land (\lnot r \lor s)  \\
    & \customeq{Def. Distr. von \land ueber \lor} & (\lnot r \land q) \lor (s \land q) \\
    & \customeq{Def. De Morgan II} & \lnot(r \lor \lnot q) \lor (s \land q) \\
    \end{aligned}
\end{equation*}

\end{solution}


\begin{problem}
    Variablenbelegungen
\end{problem}

\begin{solution}[a]
Beweise ausschließlich mit Hilfe von Argumenten über eine oder mehrere Variablenbelegungen, dass \( q \rightarrow \lnot (r \land s) \equiv \lnot q \lor (r \rightarrow \lnot s) \).

\begin{equation*}
    \begin{aligned}
    & & \llbracket q \rightarrow \lnot (r \land s) \rrbracket ^ \beta = W \\
    & \customeq{Def. Impl.} & \llbracket q \rrbracket ^ \beta = F oder \llbracket \lnot (r \land s) \rrbracket ^ \beta = W  \\
    & \customeq{Def. De Morgan I} & \llbracket q \rrbracket ^ \beta = F oder \llbracket \lnot r \lor \lnot s \rrbracket ^ \beta = W  \\
    & \customeq{Def. Impl.} & \llbracket q \rrbracket ^ \beta = F oder \llbracket r \rightarrow \lnot s \rrbracket ^ \beta = W  \\
    & \customeq{Def. \lnot} & \llbracket \lnot q \rrbracket ^ \beta = W oder \llbracket r \rightarrow \lnot s \rrbracket ^ \beta = W  \\
    & \customeq{Def. \lor} & \llbracket \lnot q \lor r \rightarrow \lnot s \rrbracket ^ \beta = W  \\
    \end{aligned}
\end{equation*}

Damit wird \( q \rightarrow \lnot (r \land s) \) genau dann zu W ausgewertet, wenn \( \lnot q \lor (r \rightarrow \lnot s) \) zu W ausgewertet wird.
Also sind die beiden Formeln äquivalent.

\end{solution}
    
\begin{solution}[b]
Beweise oder widerlege ausschließlich mit Hilfe von Argumenten über eine oder mehrere Variablenbelegungen, dass \( \lnot (\lnot q \lor (s \land r)) \lor (q \leftrightarrow (s \land r)) \) allgemeingültig ist.

Betrachte die Belegung \( \beta \) mit \( \beta (q) = F \) und \( \beta (r) = \beta (s) = W \). Dann ist
\[ \llbracket \lnot (\lnot q \lor (s \land r)) \lor (q \leftrightarrow (s \land r)) \rrbracket = F \]

Damit ist die Formel nicht allgemeingültig (da es eine Belegung gibt, unter der die Formel zu F ausgewertet wird).

\end{solution}
    
\begin{problem}
    Prädikatenlogik
\end{problem}

\begin{solution}
Beweise: \( ((\exists y . P_1(y) \rightarrow P_2(y)) \land (\forall x . P_1(x))) \rightarrow \exists z . P_2(z) \land P_1(z) \)

Annahme (A1): \( ((\exists y . P_1(y) \rightarrow P_2(y)) \land (\forall x . P_1(x))) \)

Zu Zeigen (Z1): \( \exists z . P_2(z) \land P_1(z) \)

Annahme (A2): \( \exists y . P_1(y) \rightarrow P_2(y) \)

Annahme (A3): \( \forall x . P_1(x) \)

Wähle \(  x \triangleq y \) in A3

Annahme (A4): \( P_1(y) \)

Sei x (beliebig aber fest) in A2

Annahme (A5): \( P_1(y) \rightarrow P_2(y) \)

Aus A4 und A5 folgt A6

Annahme (A6): \( P_2(y) \)

Wähle \( z \triangleq y \) in Z1

Zu Zeigen (Z3): \( P_2(y) \land P_1(y) \)

Teil 1: Zu Zeigen (Z1.1): \( P_2(y) \)

\qquad Aus A6 folgt Z1.1

Teil 2: Zu Zeigen (Z2.1): \( P_1(y) \)

\qquad Aus A4 folgt Z2.1

\end{solution}

\begin{problem}
    Widerspruch und Kontraposition
\end{problem}

\begin{solution}[a]
    Ziehe, durch die schrittweise Anwendung logischer Äquivalenzen, alle Negationen inder folgenden Formel soweit wie möglich nach Innen. Begründe jeden Schritt.

    \begin{equation*}
        \begin{aligned}
        && \lnot(\lnot(\exists x . P_1(x)) \rightarrow (\forall y . P_2(y) \land P_3(y))) \\
        & \customeq{Def. Impl.} & \lnot(\lnot \lnot(\exists x . P_1(x)) \lor (\forall y . P_2(y) \land P_3(y))) \\
        & \customeq{Def. dopp. Neg.} & \lnot((\exists x . P_1(x)) \lor (\forall y . P_2(y) \land P_3(y))) \\
        & \customeq{Def.DeMorgan2} & \lnot(\exists x . P_1(x)) \land \lnot (\forall y . P_2(y) \land P_3(y)) \\
        & \customeq{Def. log.\textrm{Äquivalenz}} & (\forall x . \lnot P_1(x)) \land \lnot (\forall y . P_2(y) \land P_3(y)) \\
        & \customeq{Def. DeMorgan1} & (\forall x . \lnot P_1(x)) \land (\lnot(\forall y . P_2(y)) \lor \lnot P_3(y)) \\
        & \customeq{Def. log.\textrm{Äquivalenz}} & (\forall x . \lnot P_1(x)) \land (\exists y . \lnot P_2(y) \lor \lnot P_3(y)) \\
        \end{aligned}
    \end{equation*}
\end{solution}

\begin{solution}[b]
    Gib an: Den ersten Schritt, d.h. die erste Zeile, eines Beweises per Widerspruch für die
    Aussage \( \lnot(\exists x . P_1(x)) \rightarrow (\forall y . P_2(y) \land P_3(y)) \)

    Widerspruchs Annahme: \( p \equiv \lnot(\lnot(\exists x . P_1(x)) \rightarrow (\forall y . P_2(y) \land P_3(y))) \rightarrow \bot \)

\end{solution}

\begin{solution}[c]
    Gib an: Den ersten Schritt, d.h. die erste Zeile, eines Beweises per Kontraposition für die
    Aussage \( \lnot(\exists x . P_1(c)) \rightarrow (\forall y . P_2(y) \land P_3(y)) \)

    Zu Zeigen: \( \lnot(\forall y . P_2(y) \land P_3(y)) \rightarrow \lnot \lnot(\exists x . P_1(x))\)

\end{solution}

\begin{problem}
    Induktion
\end{problem}

\begin{solution}
    Beweise per Induktion \( \forall n \in \mathbb N_7 . n \bmod 2 = 1\).

    Hinweis H1: \( (n+m) \bmod r = ((n \bmod r)(m \bmod r)) \bmod r \)

    Sei  
    \begin{equation*}
        \begin{aligned}
        &  P(n) \triangleq (n\bmod 2 = 1 ) \\
        \end{aligned}
    \end{equation*}

    Wir verwenden das Induktionsschema: 
    \begin{equation*}
        \begin{aligned}
        &  P(7) \land (\forall n \land \mathbb N_7 . P(n) \rightarrow P(n+10)) \rightarrow (\forall x \in \mathbb N_7 . P(x)) \\
        \end{aligned}
    \end{equation*}

    \textbf{IA (}\(P(7)\)\textbf{)}:
    \begin{equation*}
        \begin{aligned}
        & 7\bmod 2 = 1 \\
        \end{aligned}
    \end{equation*}

    Sei \( n \in \mathbb N_7 \).

    \textbf{IV (}\(P(n)\)\textbf{)}:
    \begin{equation*}
        \begin{aligned}
        &  n \bmod 2 = 1 \\
        \end{aligned}
    \end{equation*}

    \textbf{IS (}\(P(n+10)\)\textbf{)}: Zu Zeigen: \( (n + 10) \bmod 2=1 \)
    \begin{equation*}
        \begin{aligned}
        (n + 10) \bmod 2 \customeq{H1} & ((n \bmod 2)+(10 \bmod 2)) \bmod 2 \\
        = & ((n \bmod 2) + 0) \bmod 2 \\
        \customeq{IV} & (1 + 0) \bmod 2 \\
        = & 1 \bmod 2 \\
        = & 1
        \end{aligned}
    \end{equation*}

    Nach unserem Induktionsschema gilt nun \( \forall x \in \mathbb N_7 . P(x) \) was äquivalent zur ursprünglichen Aussage ist. Damit ist die Aussage bewiesen.


\end{solution}

\end{document}